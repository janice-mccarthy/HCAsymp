\documentclass[11pt]{article}
\usepackage{amsmath}

\usepackage{geometry}                % See geometry.pdf to learn the layout options. There are lots.
\geometry{letterpaper}                   % ... or a4paper or a5paper or ... 
%\geometry{landscape}                % Activate for for rotated page geometry
%\usepackage[parfill]{parskip}    % Activate to begin paragraphs with an empty line rather than an indent
\usepackage{graphicx}
\usepackage{amssymb}
\usepackage{epstopdf}
\usepackage{bbm}
%\usepackage{bibtex}
\DeclareGraphicsRule{.tif}{png}{.png}{`convert #1 `dirname #1`/`basename #1 .tif`.png}

\title{\sc Notes on the asymptotic null distribution for modified higher criticism}
\author{Janice M. McCarthy}
%\date{}                                           % Activate to display a given date or no date

\begin{document}

According to Donoho and Lin \cite{HCHM}, the higher criticism statistic is defined as:

$$\max_{0<\alpha<\alpha_0}\left(
\frac{(\textrm{Fraction Significant at } \alpha - \alpha}{\alpha (1 - \alpha)}\right)$$

In the case of $n$ independent tests, with iid test statistics $Z_i$ (not necessarily normally distributed - finite mean and variance is probably enough), the 'fraction significant at $\alpha$ is
$$ F_n(t) = \frac1n \sum_{i=1}^n \mathbbm{1}_{|Z_i|>t}$$

where $t$ is the value such that $P(|Z_i|>t) =\alpha$. For finite $n$ and $Z_1,...,Z_n$ iid, $F_n(t) ~ Bin(n,\alpha)$. If the $Z_i$ are independent but \emph{not} identically distributed, $F_n(t)$ follows a Poisson Binomial distribution. Applied to directly to p-values

$$ F_n(\alpha) = \frac1n \sum_{i=1}^n \mathbbm{1}_{p_i < \alpha}$$

If we assume that $p_i$ are iid $Unif(0,1)$ under the null, the CLT tells us that 

$$\sqrt{n}\left(\frac{F_n(\alpha) - \alpha}{\sqrt{\alpha(1-\alpha)}}\right) \xrightarrow{d} N(0,1)$$

This is identical to statistic defined in \cite{Xihong}:

$$ S(t) = \sum_{i=1}^n \mathbbm{1}_{p_i < \alpha}$$

$$HC(t) = \frac{S(t)-n t}{(\sqrt{nt(1-t)}} = n\frac{S(t)/n - t}{\sqrt{nt(1-t)}} 
=   \sqrt{n}\frac{F_n(t) - t}{\sqrt{t(1-t)}}$$

The first expression can be interpreted as a centered and scaled binomial random variable. The last expression is called the 'normalized uniform empirical process'. Clearly, it is asymptotically standard normal for each $0<t<1$

However, we want the $\sup$ (or $\max$), and this changes the asymptotics. 

$$HC^* = \max_{0<t<1} HC(t)$$

\bibliographystyle{plain}
\bibliography{refs}

\end{document} 